\section*{Abstract}

\addcontentsline{toc}{section}{Abstract}

\begin{comment}
Write a short abstract restating the main objectives of the project, your main findings and their significance. Do not exceed 100 words in the abstract. \\

The maximum length of the entire report is 50 pages, excluding appendices, bibliography and cover sheet / abstract / acknowledgements / nomenclature / table of contents / lists of figures and tables. Please note that, differently from the overall 50 page limit, the limits provided for each section (given later) are indicative. You can use Microsoft Word or other document editors such as LaTeX, InDesign, or OpenOffice, however font sizes and margins need to be very similar to this template. Please also refer to the course outline for details on the expectations regarding this document (marking criteria and rubrics).
\end{comment}

Photovoltaic modules play a central role in meeting the growing global demand for renewable energy, yet their electrical efficiency declines as operating temperature rises. Vortex generators have been proposed as a means of enhancing convective heat transfer and mitigating this thermally induced performance loss; however, the height-dependent effects of vortex generators on photovoltaic module temperature reduction remain insufficiently understood. To address this gap, forced-convection wind-tunnel experiments were conducted using cylindrical vortex generators of three heights (15~mm, 75~mm, and 150~mm), with module temperatures measured across multiple airflow velocities and analysed through MATLAB-based post-processing, energy-balance correlations, and supporting CFD analysis. It was found that centrally positioning the thermal camera artificially increased cooling due to tripod-generated Kármán vortices, and that relocating the camera to the side removed this interference and produced reliable, unaffected measurements. Ambient temperature was also found to fundamentally alter vortex-generator behaviour: the 15~mm array exhibited a reduced module-temperature-to-ambient-temperature gradient compared with the baseline (0.9 versus 1.2--1.3), leading to heating at lower ambient temperatures and a transition to cooling at higher temperatures as bottom-surface convection increased. Across all wind speeds, the 15~mm array was the most effective configuration, producing the least heating at low ambient temperatures (e.g., 1.12~\textdegree C at 1~m/s) and achieving up to 0.99~\textdegree C of cooling at higher ambient temperatures, with cooling initiating at a lower threshold than the other tested heights. Overall, the results show that vortex-generator height governs cooling performance through its influence on bottom-surface convection, revealing a clear link between inlet flow area and module temperature.

