% Conclusions should highlight the main points of your work, and stress their significance for the field and for future works to follow. 
% About \textbf{1 page} should be sufficient to concisely summarise the main conclusions.
\section{Conclusion}
This thesis investigated the influence of vortex generator height on photovoltaic module temperature reduction under forced convection. Building on the work of Schiffmann et al.~\cite{Schiffmann2023}, Chaudhury et al.~\cite{Chaudhury2024}, and Zhou et al.~\cite{Zhou2024}, this thesis revisited prior experimental observations and identified new trends that collectively advance the understanding of how vortex generators influence photovoltaic module temperatures. Through controlled wind-tunnel experiments conducted across three wind speeds and a range of ambient temperatures, the results produced a coherent set of temperature–height relationships that advance current knowledge on vortex generator assisted cooling of photovoltaic modules. During the experimental campaign, it was identified that the central positioning of the thermal camera and tripod produced unintended artificial cooling of the photovoltaic module. This observation prompted a modification to the experimental rig, wherein the thermal camera and tripod were repositioned from the central airflow region to the side of the wind tunnel. This modification ensured that tripod-induced turbulence no longer reached the module’s top surface, preventing any artificial cooling of the measured temperature. Furthermore, this finding suggests that the magnitudes of the results presented by Chaudhury et al. \cite{Chaudhury2024} and Zhou et al. \cite{Zhou2024} may require reconsideration, as their experimental setups did not account for the impact of the tripod placement.

This thesis also identified a previously unrecognised effect of ambient temperature on the photovoltaic module’s temperature behaviour when vortex generators were installed. Although ambient temperature variation had been accounted for in earlier studies, an underlying assumption persisted that the photovoltaic module’s temperature depended on ambient temperature with identical proportionality for both the vortex generator and baseline configurations, effectively treating their gradients as equal. This was not the case, revealing a notable behaviour: across all vortex‐generator heights, the array increased the module temperature at lower ambient temperatures and reduced it at higher ambient temperatures. A mathematical derivation of the energy-balance equations, supported by CFD analysis, indicates that increasing ambient temperature weakens the recirculation wakes beneath the module, enhancing bottom-surface convection. Recent baseline tests at higher ambient temperatures reveal, however, that there is a consistent ~1.4~°C reduction in baseline module temperature, suggesting that part of the apparent vortex-generator effect may reflect this shifted baseline. Interpreting the vortex-generator results therefore depends on confirming the revised baseline behaviour under higher ambient-temperature conditions.

Finally, of the vortex-generator heights tested comprehensively, the 15~mm array was found to be the most effective at reducing photovoltaic module temperature at high ambient temperatures and to cause the least heating at lower ambient temperatures. Similar to the approach used to explain the influence of ambient temperature on photovoltaic module temperature change, the energy balance equations for the 15~mm and 75~mm vortex generator arrays were derived and evaluated. This analysis identified the reduced convective heat transfer coefficient at the module’s underside as the primary mechanism behind the lower heat transfer performance of the 75~mm array relative to the 15~mm array. Further examination of the flow characteristics showed that the larger obstruction created by the 75~mm vortex generators decreased the effective inlet cross-sectional area, which in turn reduced the airflow velocity and the corresponding convective heat transfer coefficient at the bottom surface of the photovoltaic module.

This work contributes a re-evaluation of previous experimental results that informed refinements to the testing rig, an assessment of how ambient temperature affects photovoltaic module temperature under vortex-generator influence, and the identification of an optimal vortex-generator height to guide future studies. The findings advance efforts to lower module temperatures and improve solar-to-electric energy conversion, with direct relevance for rooftop systems and utility-scale solar farms. Overall, the research supports the broader transition toward cleaner and more efficient renewable-energy technologies.

Despite these contributions, several limitations should be acknowledged. Constraints inherent to the experimental rig prevented the reliable testing of the 150 mm vortex-generator array, restricting the completeness of the height-based comparison. Moreover, the limited range of vortex-generator heights examined constrained the ability to identify the configuration most effective at reducing photovoltaic module temperature. These limitations highlight the need for broader parametric testing in future work. To address these limitations, future work should incorporate upgrades to the experimental rig that enable secure mounting and reliable testing of the 150 mm vortex generator array, allowing repeated experiments for this configuration. Additionally, testing a broader range of intermediate vortex generator heights between 15 mm and 150 mm would support a more precise identification of the height that yields the greatest reduction in photovoltaic module temperature. In addition to the findings reported by Chaudhury et al.~\cite{Chaudhury2024} and Zhou et al.~\cite{Zhou2024}, the influence of the thermal camera and its associated tripod positioning raises concerns regarding the validity of the baseline test cases, as central tripod placement may have affected those measurements as well. Consequently, future work should include re-running the baseline experiments, particularly at higher ambient temperatures, to establish reliable reference data. Moreover, independent validation of the test cases conducted by Zhou et al. \cite{Zhou2024} and Chaudhury et al. \cite{Chaudhury2024} is recommended to ensure the robustness of their reported trends under corrected experimental conditions. Finally, future work should explore additional vortex-generator parameters, with one example being an assessment of roof versus underside placement, to refine optimal configurations for photovoltaic module temperature reduction.
