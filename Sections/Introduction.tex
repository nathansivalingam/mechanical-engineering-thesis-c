\section{Introduction}
\begin{comment}
This section should introduce your topic (to a reader who is an engineer, but may not be fully familiar with your topic – this is for the benefit of readers other than your supervisor, i.e. the moderator or incoming future students). Here you need to provide an overview of what the field is, what your project is, why your project is important, and very briefly how you have organized your work. It should set up the literature review – the reader now knows roughly what they are reading about and what they are looking for in the coming section(s). \\
    
The introduction might run to about \textbf{1 to 2 pages}, assuming a figure or two showing something highly useful for the reader, such as a picture of the problem (perhaps with annotation), and maybe some kind of flowchart showing the workflow.

\end{comment}

Throughout history, reliance on non-renewable energy sources has been central to the rapid advancement of society, ultimately culminating in the emergence of the modern world. Today, the global population, now exceeding eight billion, remains heavily dependent on these energy sources for survival and continued development. However, the accelerated pace of advancement, coupled with rapid population growth, has led to the exploitation of these finite resources, resulting in significant environmental damage and their depletion. Fossil fuels such as coal, oil, and natural gas are responsible for approximately 89\% of global greenhouse gas emissions, which are recognised as the primary contributors to climate change \cite{ClientEarth2025}. Furthermore, the combustion of fossil fuels accounts for roughly 70\% of global air pollution, a major factor in an estimated 7 million premature deaths annually worldwide \cite{WorldHealthOrganization2024}. Moreover, proven oil reserves are decreasing by 3-4\% each year, as the rate of global extraction surpasses the discovery of new reserves, raising significant concerns about future supply shortages \cite{ExxonMobil2025}.

To address this challenge, the large-scale development of renewable energy sources was initiated. Wind turbines harnessed wind energy, dams and rivers captured the power of flowing water, and photovoltaic (PV) modules converted solar energy into electricity. This marked a global shift in energy dependence, with populations transitioning from non-renewable to sustainable energy sources. As of 2024, renewable energy sources account for approximately 40.9\% of global electricity generation, with solar power serving as the primary driver of this growth \cite{iea2024}.\par

As reliance on renewable energy sources grew, further investigation into their performance limitations became essential. One key issue identified was the reduction in the electrical efficiency of photovoltaic modules at elevated temperatures. In their seminal 1961 paper, \textit{Detailed Balance Limit of Efficiency of p-n Junction Solar Cells}, William Shockley and Hans Queisser established the theoretical maximum efficiency of silicon-based solar cells at approximately 30\% under standard illumination conditions \cite{Shockley1961}. Moreover, the electrical efficiency of a typical photovoltaic module generally ranges from 6\% to 20\% \cite{Dubey2013}. In 2010, H.G. Teo observed the inverse relationship between the electrical efficiency of photovoltaic modules and their operating temperature, noting that higher temperatures result in further efficiency losses \cite{Teo2012}. Specifically, for every degree Celsius above the optimal temperature of 25 \textdegree C, the electrical efficiency decreases by 0.5\% \cite{MorganKingThePanels}.\par

Thus, this thesis will investigate the height effects of vortex generators (VG) on photovoltaic module temperature reduction, with the aim of determining the optimal height to maximise cooling performance. To fulfil this aim, geometric parameters of the vortex generator array were selected for investigation, including the heights and shape of the vortex generators, as well as the position and spacing of the array. An experimental testing procedure, comprising an experimental method and a post-processing approach built on the work of previous thesis students, was then established. Past results from these earlier thesis campaigns were collected and used to validate the experimental procedure, ensuring confidence in its reliability. Following this validation, different vortex-generator heights were tested to identify the most effective height configuration.

Prior to establishing an experimental procedure, it is necessary to justify the focus on vortex generators and their height effects in reducing photovoltaic module temperatures. Accordingly, this thesis begins by examining existing methods designed to lower the operating temperature of photovoltaic modules. These methods are evaluated based on their effectiveness in reducing module temperature and, in turn, improving electrical efficiency. Through this review and assessment, vortex generators are identified as a promising cooling strategy, with the influence of their height emerging as an important gap in the literature warranting further investigation.
